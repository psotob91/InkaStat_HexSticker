% Options for packages loaded elsewhere
\PassOptionsToPackage{unicode}{hyperref}
\PassOptionsToPackage{hyphens}{url}
%
\documentclass[
]{article}
\usepackage{amsmath,amssymb}
\usepackage{lmodern}
\usepackage{iftex}
\ifPDFTeX
  \usepackage[T1]{fontenc}
  \usepackage[utf8]{inputenc}
  \usepackage{textcomp} % provide euro and other symbols
\else % if luatex or xetex
  \usepackage{unicode-math}
  \defaultfontfeatures{Scale=MatchLowercase}
  \defaultfontfeatures[\rmfamily]{Ligatures=TeX,Scale=1}
\fi
% Use upquote if available, for straight quotes in verbatim environments
\IfFileExists{upquote.sty}{\usepackage{upquote}}{}
\IfFileExists{microtype.sty}{% use microtype if available
  \usepackage[]{microtype}
  \UseMicrotypeSet[protrusion]{basicmath} % disable protrusion for tt fonts
}{}
\makeatletter
\@ifundefined{KOMAClassName}{% if non-KOMA class
  \IfFileExists{parskip.sty}{%
    \usepackage{parskip}
  }{% else
    \setlength{\parindent}{0pt}
    \setlength{\parskip}{6pt plus 2pt minus 1pt}}
}{% if KOMA class
  \KOMAoptions{parskip=half}}
\makeatother
\usepackage{xcolor}
\IfFileExists{xurl.sty}{\usepackage{xurl}}{} % add URL line breaks if available
\IfFileExists{bookmark.sty}{\usepackage{bookmark}}{\usepackage{hyperref}}
\hypersetup{
  pdftitle={INKASTATS hexsticker},
  hidelinks,
  pdfcreator={LaTeX via pandoc}}
\urlstyle{same} % disable monospaced font for URLs
\usepackage[margin=1in]{geometry}
\usepackage{color}
\usepackage{fancyvrb}
\newcommand{\VerbBar}{|}
\newcommand{\VERB}{\Verb[commandchars=\\\{\}]}
\DefineVerbatimEnvironment{Highlighting}{Verbatim}{commandchars=\\\{\}}
% Add ',fontsize=\small' for more characters per line
\usepackage{framed}
\definecolor{shadecolor}{RGB}{248,248,248}
\newenvironment{Shaded}{\begin{snugshade}}{\end{snugshade}}
\newcommand{\AlertTok}[1]{\textcolor[rgb]{0.94,0.16,0.16}{#1}}
\newcommand{\AnnotationTok}[1]{\textcolor[rgb]{0.56,0.35,0.01}{\textbf{\textit{#1}}}}
\newcommand{\AttributeTok}[1]{\textcolor[rgb]{0.77,0.63,0.00}{#1}}
\newcommand{\BaseNTok}[1]{\textcolor[rgb]{0.00,0.00,0.81}{#1}}
\newcommand{\BuiltInTok}[1]{#1}
\newcommand{\CharTok}[1]{\textcolor[rgb]{0.31,0.60,0.02}{#1}}
\newcommand{\CommentTok}[1]{\textcolor[rgb]{0.56,0.35,0.01}{\textit{#1}}}
\newcommand{\CommentVarTok}[1]{\textcolor[rgb]{0.56,0.35,0.01}{\textbf{\textit{#1}}}}
\newcommand{\ConstantTok}[1]{\textcolor[rgb]{0.00,0.00,0.00}{#1}}
\newcommand{\ControlFlowTok}[1]{\textcolor[rgb]{0.13,0.29,0.53}{\textbf{#1}}}
\newcommand{\DataTypeTok}[1]{\textcolor[rgb]{0.13,0.29,0.53}{#1}}
\newcommand{\DecValTok}[1]{\textcolor[rgb]{0.00,0.00,0.81}{#1}}
\newcommand{\DocumentationTok}[1]{\textcolor[rgb]{0.56,0.35,0.01}{\textbf{\textit{#1}}}}
\newcommand{\ErrorTok}[1]{\textcolor[rgb]{0.64,0.00,0.00}{\textbf{#1}}}
\newcommand{\ExtensionTok}[1]{#1}
\newcommand{\FloatTok}[1]{\textcolor[rgb]{0.00,0.00,0.81}{#1}}
\newcommand{\FunctionTok}[1]{\textcolor[rgb]{0.00,0.00,0.00}{#1}}
\newcommand{\ImportTok}[1]{#1}
\newcommand{\InformationTok}[1]{\textcolor[rgb]{0.56,0.35,0.01}{\textbf{\textit{#1}}}}
\newcommand{\KeywordTok}[1]{\textcolor[rgb]{0.13,0.29,0.53}{\textbf{#1}}}
\newcommand{\NormalTok}[1]{#1}
\newcommand{\OperatorTok}[1]{\textcolor[rgb]{0.81,0.36,0.00}{\textbf{#1}}}
\newcommand{\OtherTok}[1]{\textcolor[rgb]{0.56,0.35,0.01}{#1}}
\newcommand{\PreprocessorTok}[1]{\textcolor[rgb]{0.56,0.35,0.01}{\textit{#1}}}
\newcommand{\RegionMarkerTok}[1]{#1}
\newcommand{\SpecialCharTok}[1]{\textcolor[rgb]{0.00,0.00,0.00}{#1}}
\newcommand{\SpecialStringTok}[1]{\textcolor[rgb]{0.31,0.60,0.02}{#1}}
\newcommand{\StringTok}[1]{\textcolor[rgb]{0.31,0.60,0.02}{#1}}
\newcommand{\VariableTok}[1]{\textcolor[rgb]{0.00,0.00,0.00}{#1}}
\newcommand{\VerbatimStringTok}[1]{\textcolor[rgb]{0.31,0.60,0.02}{#1}}
\newcommand{\WarningTok}[1]{\textcolor[rgb]{0.56,0.35,0.01}{\textbf{\textit{#1}}}}
\usepackage{graphicx}
\makeatletter
\def\maxwidth{\ifdim\Gin@nat@width>\linewidth\linewidth\else\Gin@nat@width\fi}
\def\maxheight{\ifdim\Gin@nat@height>\textheight\textheight\else\Gin@nat@height\fi}
\makeatother
% Scale images if necessary, so that they will not overflow the page
% margins by default, and it is still possible to overwrite the defaults
% using explicit options in \includegraphics[width, height, ...]{}
\setkeys{Gin}{width=\maxwidth,height=\maxheight,keepaspectratio}
% Set default figure placement to htbp
\makeatletter
\def\fps@figure{htbp}
\makeatother
\setlength{\emergencystretch}{3em} % prevent overfull lines
\providecommand{\tightlist}{%
  \setlength{\itemsep}{0pt}\setlength{\parskip}{0pt}}
\setcounter{secnumdepth}{-\maxdimen} % remove section numbering
\ifLuaTeX
  \usepackage{selnolig}  % disable illegal ligatures
\fi

\title{INKASTATS hexsticker}
\author{}
\date{\vspace{-2.5em}}

\begin{document}
\maketitle

\hypertarget{configurar-formato-de-letras}{%
\subsection{Configurar formato de
letras}\label{configurar-formato-de-letras}}

\begin{Shaded}
\begin{Highlighting}[]
\DocumentationTok{\#\# Loading Google fonts (http://www.google.com/fonts)}
\FunctionTok{font\_add\_google}\NormalTok{(}\StringTok{"Poiret One"}\NormalTok{, }\StringTok{"poiret"}\NormalTok{)}
\FunctionTok{font\_add\_google}\NormalTok{(}\StringTok{"Raleway"}\NormalTok{, }\StringTok{"raleway"}\NormalTok{)}
\FunctionTok{font\_add\_google}\NormalTok{(}\StringTok{"Roboto Slab"}\NormalTok{, }\StringTok{"roboto\_slab"}\NormalTok{)}
\FunctionTok{font\_add\_google}\NormalTok{(}\StringTok{"Gochi Hand"}\NormalTok{, }\StringTok{"gochi"}\NormalTok{)}

\DocumentationTok{\#\# loading local font }
\FunctionTok{font\_add}\NormalTok{(}\AttributeTok{family =} \StringTok{"broadway"}\NormalTok{, }\AttributeTok{regular =} \StringTok{"BROADW.TTF"}\NormalTok{)}

\FunctionTok{font\_add}\NormalTok{(}\StringTok{"trebuchet"}\NormalTok{,}
         \AttributeTok{regular=}\StringTok{"C:}\SpecialCharTok{\textbackslash{}\textbackslash{}}\StringTok{Windows}\SpecialCharTok{\textbackslash{}\textbackslash{}}\StringTok{Fonts}\SpecialCharTok{\textbackslash{}\textbackslash{}}\StringTok{trebuc.ttf"}\NormalTok{)}

\FunctionTok{font\_add}\NormalTok{(}\StringTok{"gill"}\NormalTok{,}
         \AttributeTok{regular=}\StringTok{"C:}\SpecialCharTok{\textbackslash{}\textbackslash{}}\StringTok{Windows}\SpecialCharTok{\textbackslash{}\textbackslash{}}\StringTok{Fonts}\SpecialCharTok{\textbackslash{}\textbackslash{}}\StringTok{GILSANUB.TTF"}\NormalTok{)}

\DocumentationTok{\#\# Automatically use showtext to render text for future devices}
\FunctionTok{showtext\_auto}\NormalTok{()}
\end{Highlighting}
\end{Shaded}

\hypertarget{configurar-colores}{%
\subsubsection{configurar colores}\label{configurar-colores}}

\begin{Shaded}
\begin{Highlighting}[]
\NormalTok{uvic\_blue      }\OtherTok{\textless{}{-}} \StringTok{"\#005493"}    \CommentTok{\#RGB: 0{-}84{-}147}
\NormalTok{uvic\_yellow    }\OtherTok{\textless{}{-}} \StringTok{"\#F5AA1C"}    \CommentTok{\#RGB: 245{-}170{-}28  }
\NormalTok{uvic\_red       }\OtherTok{\textless{}{-}} \StringTok{"\#C63527"}    \CommentTok{\#RGB: 198{-}53{-}39}
\NormalTok{uvic\_blue\_dark }\OtherTok{\textless{}{-}} \StringTok{"\#002754"}  \CommentTok{\#RGB: 0{-}39{-}84}
\NormalTok{uvic\_white     }\OtherTok{\textless{}{-}} \StringTok{"\#ffffff"} \CommentTok{\#RGB: 255{-}255{-}255}
\NormalTok{uvic\_golden    }\OtherTok{\textless{}{-}} \StringTok{"\#eecd24"} \CommentTok{\#RGB: 238{-}205{-}36}
\NormalTok{uvic\_black     }\OtherTok{\textless{}{-}} \StringTok{"\#000000"} \CommentTok{\#RGB: 0{-}0{-}0}
\NormalTok{uvic\_borde     }\OtherTok{\textless{}{-}} \StringTok{"\#5b5b5b"}
\NormalTok{uvic\_borde2    }\OtherTok{\textless{}{-}} \StringTok{"\#424242"}
\NormalTok{borde3         }\OtherTok{\textless{}{-}} \StringTok{"\#20324B"}
\NormalTok{borde4         }\OtherTok{\textless{}{-}} \StringTok{"\#363636"}
\NormalTok{uvic\_fondo     }\OtherTok{\textless{}{-}} \StringTok{"\#d6d6d6"}
\NormalTok{url\_color      }\OtherTok{\textless{}{-}}\NormalTok{ uvic\_yellow}
\end{Highlighting}
\end{Shaded}

\hypertarget{quipu-con-fondo-de-aestethic-de-ggplot2}{%
\subsection{Quipu con fondo de aestethic de
ggplot2}\label{quipu-con-fondo-de-aestethic-de-ggplot2}}

\begin{Shaded}
\begin{Highlighting}[]
\CommentTok{\#Importar quipu y guardarlo como objeto raster}
\NormalTok{quipu }\OtherTok{\textless{}{-}} \FunctionTok{readPNG}\NormalTok{(}\StringTok{"../images\_sources/quipu\_fn.png"}\NormalTok{)}

\NormalTok{quipu\_raster }\OtherTok{\textless{}{-}} \FunctionTok{rasterGrob}\NormalTok{(     }\CommentTok{\#Para agregar el quipu solo}
\NormalTok{  quipu, }
  \AttributeTok{width =} \FloatTok{0.7}\NormalTok{,}
  \AttributeTok{height =} \FloatTok{0.5}\NormalTok{, }
  \AttributeTok{x =} \FloatTok{0.5}\NormalTok{, }
  \AttributeTok{y =} \FloatTok{0.6}\NormalTok{,}
  \AttributeTok{interpolate =}\NormalTok{ T)}
\end{Highlighting}
\end{Shaded}

\begin{Shaded}
\begin{Highlighting}[]
\CommentTok{\# Crear data simulada para fondo}
\NormalTok{xpos }\OtherTok{\textless{}{-}} \DecValTok{1}\SpecialCharTok{:}\DecValTok{5}
\NormalTok{ypos }\OtherTok{\textless{}{-}}\NormalTok{ xpos}\SpecialCharTok{**}\DecValTok{2}
  
\NormalTok{data\_frame }\OtherTok{=} \FunctionTok{data.frame}\NormalTok{(}\AttributeTok{xpos =}\NormalTok{ xpos,}
                        \AttributeTok{ypos =}\NormalTok{ ypos)}
  
\FunctionTok{print}\NormalTok{(}\StringTok{"Data points"}\NormalTok{)}
\end{Highlighting}
\end{Shaded}

\begin{verbatim}
## [1] "Data points"
\end{verbatim}

\begin{Shaded}
\begin{Highlighting}[]
\FunctionTok{print}\NormalTok{(data\_frame)}
\end{Highlighting}
\end{Shaded}

\begin{verbatim}
##   xpos ypos
## 1    1    1
## 2    2    4
## 3    3    9
## 4    4   16
## 5    5   25
\end{verbatim}

\begin{Shaded}
\begin{Highlighting}[]
\CommentTok{\# Crear fondo, traslaparlo con quipu y guardarlo como objeto ggplo2}
\FunctionTok{qplot}\NormalTok{(xpos, ypos, }\AttributeTok{geom =} \StringTok{"blank"}\NormalTok{) }\SpecialCharTok{+} 
  \FunctionTok{annotation\_custom}\NormalTok{(quipu\_raster, }
                    \AttributeTok{xmin =} \SpecialCharTok{{-}}\ConstantTok{Inf}\NormalTok{, }
                    \AttributeTok{xmax =} \ConstantTok{Inf}\NormalTok{, }
                    \AttributeTok{ymin =} \SpecialCharTok{{-}}\ConstantTok{Inf}\NormalTok{, }
                    \AttributeTok{ymax =} \ConstantTok{Inf}\NormalTok{) }\SpecialCharTok{+} 
  \FunctionTok{scale\_x\_continuous}\NormalTok{(}\AttributeTok{breaks =} \FunctionTok{seq}\NormalTok{(}\DecValTok{1}\NormalTok{, }\DecValTok{5}\NormalTok{, }\AttributeTok{by =} \DecValTok{1}\NormalTok{), }\AttributeTok{minor\_breaks =} \ConstantTok{NULL}\NormalTok{) }\SpecialCharTok{+}
  \FunctionTok{scale\_y\_continuous}\NormalTok{(}\AttributeTok{breaks =} \FunctionTok{seq}\NormalTok{(}\DecValTok{0}\NormalTok{, }\DecValTok{25}\NormalTok{, }\AttributeTok{by =} \DecValTok{5}\NormalTok{), }\AttributeTok{minor\_breaks =} \ConstantTok{NULL}\NormalTok{) }\SpecialCharTok{+}
  \FunctionTok{theme}\NormalTok{(}\AttributeTok{panel.background =} \FunctionTok{element\_rect}\NormalTok{(}\AttributeTok{fill =} \StringTok{"\#EBEBEB"}\NormalTok{)) }\OtherTok{{-}\textgreater{}}\NormalTok{ gg}
  \CommentTok{\#theme\_gray() {-}\textgreater{} gg2}
  \CommentTok{\#theme\_light() {-}\textgreater{} gg2}
\end{Highlighting}
\end{Shaded}

\begin{Shaded}
\begin{Highlighting}[]
\FunctionTok{plot}\NormalTok{(gg)}
\end{Highlighting}
\end{Shaded}

\includegraphics{code_hexsticker_files/figure-latex/unnamed-chunk-6-1.pdf}

\hypertarget{logo-oficial---fondo-blanco}{%
\subsection{Logo oficial - fondo
blanco}\label{logo-oficial---fondo-blanco}}

\begin{Shaded}
\begin{Highlighting}[]
\NormalTok{Logo\_hex }\OtherTok{\textless{}{-}}\NormalTok{ hexSticker}\SpecialCharTok{::}\FunctionTok{sticker}\NormalTok{(}
    \DocumentationTok{\#\#Opciones de la imagen}
    \AttributeTok{subplot              =}\NormalTok{ gg, }\CommentTok{\#la imagen que usaremos}
    \CommentTok{\#s\_x                  = 1.00, s\_y = 1.25, \#posición relativa x e y de la imagen, 1=centro}
    \CommentTok{\#s\_width              = 1.55, \#ancho de la imagen}
    \CommentTok{\#s\_height             = 1.52, \#alto de la imagen}
    
    \AttributeTok{s\_x                  =} \FloatTok{0.90}\NormalTok{, }
    \AttributeTok{s\_y                  =} \FloatTok{1.00}\NormalTok{, }\CommentTok{\#posición relativa x e y de la imagen, 1=centro}
    \AttributeTok{s\_width              =} \FloatTok{2.5}\NormalTok{, }\CommentTok{\#ancho de la imagen}
    \AttributeTok{s\_height             =} \FloatTok{2.5}\NormalTok{, }\CommentTok{\#alto de la imagen}

    \DocumentationTok{\#\#Opciones para el título}
    \AttributeTok{package              =} \StringTok{"InkaStats"}\NormalTok{, }\CommentTok{\#el nombre del logo}
    \AttributeTok{p\_size               =} \DecValTok{40}\NormalTok{, }\CommentTok{\#Tamaño de la fuente}
    \AttributeTok{p\_family             =} \StringTok{"roboto\_slab"}\NormalTok{, }\CommentTok{\#define fuente de letra}
    \AttributeTok{p\_color              =}\NormalTok{ uvic\_black, }\CommentTok{\#para color de fuente}
    \AttributeTok{p\_x                  =} \DecValTok{1}\NormalTok{, }\AttributeTok{p\_y =} \FloatTok{0.6}\NormalTok{, }\CommentTok{\#para posición del título}
    
    \DocumentationTok{\#\#Opciones generales}
    \CommentTok{\# filename           = "Logo2 .png", \#fichero de salida}
    \AttributeTok{dpi                  =} \DecValTok{600}\NormalTok{, }\CommentTok{\#Resolución}
    
    \DocumentationTok{\#\#Opciones del spotlight}
    \AttributeTok{spotlight            =}\NormalTok{ F, }\CommentTok{\#agregar un brillo como reflector }
    \AttributeTok{l\_x                  =} \DecValTok{1}\NormalTok{, }\CommentTok{\#posición relativa}
    \AttributeTok{l\_y                  =} \FloatTok{1.2}\NormalTok{, }\CommentTok{\#posición relativa}
    \AttributeTok{l\_alpha              =} \FloatTok{0.1}\NormalTok{, }\CommentTok{\#Transparencia del spotlight}
    \AttributeTok{l\_width              =} \DecValTok{5}\NormalTok{, }\CommentTok{\#Amplitud del spotlight}
    
    \DocumentationTok{\#\#Fondo y borde}
    \AttributeTok{h\_fill               =}\NormalTok{ uvic\_fondo, }\CommentTok{\#color de fondo}
    \AttributeTok{h\_color              =}\NormalTok{ borde3, }\CommentTok{\#color de borde del hex}
    \AttributeTok{h\_size               =} \FloatTok{1.2}\NormalTok{, }\CommentTok{\#ancho de borde}
    \AttributeTok{white\_around\_sticker =}\NormalTok{ T, }\CommentTok{\#recortar el borde exterior del hex}
    
    \DocumentationTok{\#\#Opciones de la URL}
    \AttributeTok{url                  =} \StringTok{"Data Science Solutions SAC"}\NormalTok{, }\CommentTok{\#lo que irá abajo          }
    \AttributeTok{u\_color              =}\NormalTok{ uvic\_black, }\CommentTok{\#color de fuente}
    \AttributeTok{u\_size               =} \DecValTok{7}\NormalTok{, }\CommentTok{\#tamaño de fuente}
    \AttributeTok{u\_angle              =} \DecValTok{30}\NormalTok{, }\CommentTok{\#Ángulo de la URL}
    \AttributeTok{u\_x                  =} \DecValTok{1}\NormalTok{, }
    \AttributeTok{u\_y                  =} \FloatTok{0.05}\NormalTok{, }
    \AttributeTok{u\_family             =} \StringTok{"roboto\_slab"}\NormalTok{, }
    
    \CommentTok{\# Save file}
    \AttributeTok{filename             =} \StringTok{"../stickers\_png/logo{-}inkastats{-}official.png"}
    
\NormalTok{  )  }

\CommentTok{\# save\_sticker("Logo\_InkaStats.png", }
\CommentTok{\#              Logo\_hex, }
\CommentTok{\#              dpi         = 600)}

\CommentTok{\#Posición de la URL. Pos1: ang=30 x=0.3 y=1.5, Pos2: ang={-}30 x=1.25 y=1.75, Pos3: ang={-}90 x=0.2 y=1.45, Pos4: ang=90 x=1.8 y=0.85, Pos5: ang={-}30 x=0.5 y=0.35 Pos6: x=1 y=0.1}

\FunctionTok{plot}\NormalTok{(Logo\_hex)}
\end{Highlighting}
\end{Shaded}

\includegraphics{code_hexsticker_files/figure-latex/unnamed-chunk-7-1.pdf}

\hypertarget{logo-oficial---fondo-transparente}{%
\subsection{Logo oficial - fondo
transparente}\label{logo-oficial---fondo-transparente}}

\begin{Shaded}
\begin{Highlighting}[]
\CommentTok{\# Extrayendo fondo}
\NormalTok{logo }\OtherTok{\textless{}{-}} \FunctionTok{image\_read}\NormalTok{(}\StringTok{"../stickers\_png/logo{-}inkastats{-}official.png"}\NormalTok{)}

\NormalTok{logo }\SpecialCharTok{\%\textgreater{}\%} 
  \FunctionTok{image\_transparent}\NormalTok{(}\StringTok{\textquotesingle{}white\textquotesingle{}}\NormalTok{) }\SpecialCharTok{\%\textgreater{}\%} 
  \FunctionTok{image\_write}\NormalTok{(}\StringTok{"../stickers\_png/logo{-}inkastats{-}official{-}transparent.png"}\NormalTok{, }\StringTok{"png"}\NormalTok{)}
\end{Highlighting}
\end{Shaded}

\hypertarget{logos-no-oficiales}{%
\subsection{Logos no oficiales}\label{logos-no-oficiales}}

\hypertarget{con-otro-tipo-de-fondo}{%
\subsubsection{con otro tipo de fondo}\label{con-otro-tipo-de-fondo}}

\begin{Shaded}
\begin{Highlighting}[]
\CommentTok{\# Crear fondo, traslaparlo con quipu y guardarlo como objeto ggplo2}
\FunctionTok{qplot}\NormalTok{(xpos, ypos, }\AttributeTok{geom =} \StringTok{"blank"}\NormalTok{) }\SpecialCharTok{+} 
  \FunctionTok{annotation\_custom}\NormalTok{(quipu\_raster, }
                    \AttributeTok{xmin =} \SpecialCharTok{{-}}\ConstantTok{Inf}\NormalTok{, }
                    \AttributeTok{xmax =} \ConstantTok{Inf}\NormalTok{, }
                    \AttributeTok{ymin =} \SpecialCharTok{{-}}\ConstantTok{Inf}\NormalTok{, }
                    \AttributeTok{ymax =} \ConstantTok{Inf}\NormalTok{) }\SpecialCharTok{+} 
  \FunctionTok{scale\_x\_continuous}\NormalTok{(}\AttributeTok{breaks =} \FunctionTok{seq}\NormalTok{(}\DecValTok{1}\NormalTok{, }\DecValTok{5}\NormalTok{, }\AttributeTok{by =} \DecValTok{1}\NormalTok{), }\AttributeTok{minor\_breaks =} \ConstantTok{NULL}\NormalTok{) }\SpecialCharTok{+}
  \FunctionTok{scale\_y\_continuous}\NormalTok{(}\AttributeTok{breaks =} \FunctionTok{seq}\NormalTok{(}\DecValTok{0}\NormalTok{, }\DecValTok{25}\NormalTok{, }\AttributeTok{by =} \DecValTok{5}\NormalTok{), }\AttributeTok{minor\_breaks =} \ConstantTok{NULL}\NormalTok{) }\SpecialCharTok{+}
  \CommentTok{\#theme(panel.background = element\_rect(fill = "\#EBEBEB")) {-}\textgreater{} gg}
  \CommentTok{\#theme\_gray() {-}\textgreater{} gg2}
  \FunctionTok{theme\_light}\NormalTok{() }\OtherTok{{-}\textgreater{}}\NormalTok{ gg2}
  
\NormalTok{Logo\_hex2 }\OtherTok{\textless{}{-}}\NormalTok{ hexSticker}\SpecialCharTok{::}\FunctionTok{sticker}\NormalTok{(}
    \DocumentationTok{\#\#Opciones de la imagen}
    \AttributeTok{subplot              =}\NormalTok{ gg2, }\CommentTok{\#la imagen que usaremos}
    \CommentTok{\#s\_x                  = 1.00, s\_y = 1.25, \#posición relativa x e y de la imagen, 1=centro}
    \CommentTok{\#s\_width              = 1.55, \#ancho de la imagen}
    \CommentTok{\#s\_height             = 1.52, \#alto de la imagen}
    
    \AttributeTok{s\_x                  =} \FloatTok{0.90}\NormalTok{, }
    \AttributeTok{s\_y                  =} \FloatTok{1.00}\NormalTok{, }\CommentTok{\#posición relativa x e y de la imagen, 1=centro}
    \AttributeTok{s\_width              =} \FloatTok{2.5}\NormalTok{, }\CommentTok{\#ancho de la imagen}
    \AttributeTok{s\_height             =} \FloatTok{2.5}\NormalTok{, }\CommentTok{\#alto de la imagen}

    \DocumentationTok{\#\#Opciones para el título}
    \AttributeTok{package              =} \StringTok{"InkaStats"}\NormalTok{, }\CommentTok{\#el nombre del logo}
    \AttributeTok{p\_size               =} \DecValTok{40}\NormalTok{, }\CommentTok{\#Tamaño de la fuente}
    \AttributeTok{p\_family             =} \StringTok{"roboto\_slab"}\NormalTok{, }\CommentTok{\#define fuente de letra}
    \AttributeTok{p\_color              =}\NormalTok{ uvic\_black, }\CommentTok{\#para color de fuente}
    \AttributeTok{p\_x                  =} \DecValTok{1}\NormalTok{, }\AttributeTok{p\_y =} \FloatTok{0.6}\NormalTok{, }\CommentTok{\#para posición del título}
    
    \DocumentationTok{\#\#Opciones generales}
    \CommentTok{\# filename           = "Logo2 .png", \#fichero de salida}
    \AttributeTok{dpi                  =} \DecValTok{600}\NormalTok{, }\CommentTok{\#Resolución}
    
    \DocumentationTok{\#\#Opciones del spotlight}
    \AttributeTok{spotlight            =}\NormalTok{ F, }\CommentTok{\#agregar un brillo como reflector }
    \AttributeTok{l\_x                  =} \DecValTok{1}\NormalTok{, }\CommentTok{\#posición relativa}
    \AttributeTok{l\_y                  =} \FloatTok{1.2}\NormalTok{, }\CommentTok{\#posición relativa}
    \AttributeTok{l\_alpha              =} \FloatTok{0.1}\NormalTok{, }\CommentTok{\#Transparencia del spotlight}
    \AttributeTok{l\_width              =} \DecValTok{5}\NormalTok{, }\CommentTok{\#Amplitud del spotlight}
    
    \DocumentationTok{\#\#Fondo y borde}
    \AttributeTok{h\_fill               =}\NormalTok{ uvic\_fondo, }\CommentTok{\#color de fondo}
    \AttributeTok{h\_color              =}\NormalTok{ borde3, }\CommentTok{\#color de borde del hex}
    \AttributeTok{h\_size               =} \FloatTok{1.2}\NormalTok{, }\CommentTok{\#ancho de borde}
    \AttributeTok{white\_around\_sticker =}\NormalTok{ T, }\CommentTok{\#recortar el borde exterior del hex}
    
    \DocumentationTok{\#\#Opciones de la URL}
    \AttributeTok{url                  =} \StringTok{"Data Science Solutions SAC"}\NormalTok{, }\CommentTok{\#lo que irá abajo          }
    \AttributeTok{u\_color              =}\NormalTok{ uvic\_black, }\CommentTok{\#color de fuente}
    \AttributeTok{u\_size               =} \DecValTok{7}\NormalTok{, }\CommentTok{\#tamaño de fuente}
    \AttributeTok{u\_angle              =} \DecValTok{30}\NormalTok{, }\CommentTok{\#Ángulo de la URL}
    \AttributeTok{u\_x                  =} \DecValTok{1}\NormalTok{, }
    \AttributeTok{u\_y                  =} \FloatTok{0.05}\NormalTok{, }
    \AttributeTok{u\_family             =} \StringTok{"roboto\_slab"}\NormalTok{, }
    
    \CommentTok{\# Save file}
    \AttributeTok{filename             =} \StringTok{"../stickers\_png/logo{-}inkastats{-}noofficial{-}other.png"}
    
\NormalTok{  )  }

\CommentTok{\# save\_sticker("Logo\_InkaStats.png", }
\CommentTok{\#              Logo\_hex, }
\CommentTok{\#              dpi         = 600)}

\CommentTok{\#Posición de la URL. Pos1: ang=30 x=0.3 y=1.5, Pos2: ang={-}30 x=1.25 y=1.75, Pos3: ang={-}90 x=0.2 y=1.45, Pos4: ang=90 x=1.8 y=0.85, Pos5: ang={-}30 x=0.5 y=0.35 Pos6: x=1 y=0.1}

\FunctionTok{plot}\NormalTok{(Logo\_hex2)}
\end{Highlighting}
\end{Shaded}

\includegraphics{code_hexsticker_files/figure-latex/unnamed-chunk-9-1.pdf}

\hypertarget{con-efectos}{%
\subsubsection{Con efectos}\label{con-efectos}}

\begin{Shaded}
\begin{Highlighting}[]
\NormalTok{logo }\SpecialCharTok{\%\textgreater{}\%} 
  \FunctionTok{image\_charcoal}\NormalTok{() }\SpecialCharTok{\%\textgreater{}\%} 
  \FunctionTok{image\_write}\NormalTok{(}\StringTok{"../stickers\_png/logo{-}inkastats{-}noofficial{-}charcoal.png"}\NormalTok{, }\StringTok{"png"}\NormalTok{)}
\end{Highlighting}
\end{Shaded}

\begin{Shaded}
\begin{Highlighting}[]
\NormalTok{logo }\SpecialCharTok{\%\textgreater{}\%} 
  \FunctionTok{image\_implode}\NormalTok{() }\SpecialCharTok{\%\textgreater{}\%} 
  \FunctionTok{image\_write}\NormalTok{(}\StringTok{"../stickers\_png/logo{-}inkastats{-}noofficial{-}implode.png"}\NormalTok{, }\StringTok{"png"}\NormalTok{)}
\end{Highlighting}
\end{Shaded}

\begin{Shaded}
\begin{Highlighting}[]
\NormalTok{logo }\SpecialCharTok{\%\textgreater{}\%} 
  \FunctionTok{image\_oilpaint}\NormalTok{() }\SpecialCharTok{\%\textgreater{}\%} 
  \FunctionTok{image\_write}\NormalTok{(}\StringTok{"../stickers\_png/logo{-}inkastats{-}noofficial{-}oilpaint.png"}\NormalTok{, }\StringTok{"png"}\NormalTok{)}
\end{Highlighting}
\end{Shaded}

\begin{Shaded}
\begin{Highlighting}[]
\NormalTok{logo }\SpecialCharTok{\%\textgreater{}\%} 
  \FunctionTok{image\_edge}\NormalTok{() }\SpecialCharTok{\%\textgreater{}\%} 
  \FunctionTok{image\_write}\NormalTok{(}\StringTok{"../stickers\_png/logo{-}inkastats{-}noofficial{-}edge.png"}\NormalTok{, }\StringTok{"png"}\NormalTok{)}
\end{Highlighting}
\end{Shaded}

\hypertarget{gift-animados}{%
\subsubsection{Gift animados}\label{gift-animados}}

\begin{Shaded}
\begin{Highlighting}[]
\NormalTok{logo1 }\OtherTok{\textless{}{-}} \FunctionTok{image\_read}\NormalTok{(}\StringTok{"../stickers\_png/logo{-}inkastats{-}noofficial{-}charcoal.png"}\NormalTok{)}
\NormalTok{logo2 }\OtherTok{\textless{}{-}} \FunctionTok{image\_read}\NormalTok{(}\StringTok{"../stickers\_png/logo{-}inkastats{-}noofficial{-}oilpaint.png"}\NormalTok{)}
\NormalTok{logos }\OtherTok{\textless{}{-}} \FunctionTok{c}\NormalTok{(logo1, logo2)}

\NormalTok{(animation1 }\OtherTok{\textless{}{-}} \FunctionTok{image\_animate}\NormalTok{(logos, }\AttributeTok{fps =} \DecValTok{4}\NormalTok{)) }
\end{Highlighting}
\end{Shaded}

\includegraphics{code_hexsticker_files/figure-latex/unnamed-chunk-14-1.gif}

\begin{Shaded}
\begin{Highlighting}[]
\NormalTok{animation1 }\SpecialCharTok{\%\textgreater{}\%} 
  \FunctionTok{image\_write}\NormalTok{(}\StringTok{"../gifs/logo{-}charcoal{-}oilpaint.gif"}\NormalTok{)}
\end{Highlighting}
\end{Shaded}

\begin{Shaded}
\begin{Highlighting}[]
\NormalTok{logo1 }\OtherTok{\textless{}{-}} \FunctionTok{image\_read}\NormalTok{(}\StringTok{"../stickers\_png/logo{-}inkastats{-}noofficial{-}other.png"}\NormalTok{)}
\NormalTok{logo2 }\OtherTok{\textless{}{-}} \FunctionTok{image\_read}\NormalTok{(}\StringTok{"../stickers\_png/logo{-}inkastats{-}official.png"}\NormalTok{)}
\NormalTok{logos }\OtherTok{\textless{}{-}} \FunctionTok{c}\NormalTok{(logo1, logo2)}

\NormalTok{(animation2 }\OtherTok{\textless{}{-}} \FunctionTok{image\_animate}\NormalTok{(logos, }\AttributeTok{fps =} \DecValTok{4}\NormalTok{)) }
\end{Highlighting}
\end{Shaded}

\includegraphics{code_hexsticker_files/figure-latex/unnamed-chunk-15-1.gif}

\begin{Shaded}
\begin{Highlighting}[]
\NormalTok{animation2 }\SpecialCharTok{\%\textgreater{}\%} 
  \FunctionTok{image\_write}\NormalTok{(}\StringTok{"../gifs/logo{-}official{-}nonofficial.gif"}\NormalTok{)}
\end{Highlighting}
\end{Shaded}

\begin{Shaded}
\begin{Highlighting}[]
\NormalTok{logo1 }\OtherTok{\textless{}{-}} \FunctionTok{image\_read}\NormalTok{(}\StringTok{"../stickers\_png/logo{-}inkastats{-}noofficial{-}other.png"}\NormalTok{)}
\NormalTok{logo2 }\OtherTok{\textless{}{-}} \FunctionTok{image\_read}\NormalTok{(}\StringTok{"../stickers\_png/logo{-}inkastats{-}noofficial{-}oilpaint.png"}\NormalTok{)}
\NormalTok{logo3 }\OtherTok{\textless{}{-}} \FunctionTok{image\_read}\NormalTok{(}\StringTok{"../stickers\_png/logo{-}inkastats{-}official.png"}\NormalTok{)}
\NormalTok{logos }\OtherTok{\textless{}{-}} \FunctionTok{c}\NormalTok{(logo1, logo2, logo3)}

\NormalTok{(animation2 }\OtherTok{\textless{}{-}} \FunctionTok{image\_animate}\NormalTok{(logos, }\AttributeTok{fps =} \DecValTok{4}\NormalTok{)) }
\end{Highlighting}
\end{Shaded}

\includegraphics{code_hexsticker_files/figure-latex/unnamed-chunk-16-1.gif}

\begin{Shaded}
\begin{Highlighting}[]
\NormalTok{animation2 }\SpecialCharTok{\%\textgreater{}\%} 
  \FunctionTok{image\_write}\NormalTok{(}\StringTok{"../gifs/logo{-}official{-}oilpaint{-}nonofficial.gif"}\NormalTok{)}
\end{Highlighting}
\end{Shaded}


\end{document}
